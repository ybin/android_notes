\documentclass{article}

%%% fontenc
%\usepackage{fontspec,xunicode,xltxtra}
%\setmainfont{Times New Roman}
%\setsansfont{Source Sans Pro}
%\setmonofont{Source Sans Pro}

%%% xeCJK
\usepackage{xeCJK}
\setCJKmainfont[BoldFont=Adobe Heiti Std]{Adobe Song Std}
\setCJKsansfont[BoldFont=Adobe Heiti Std]{Adobe Song Std}
\setCJKmonofont[BoldFont=Adobe Heiti Std]{Adobe Song Std}
\XeTeXlinebreaklocale "zh"
\XeTeXlinebreakskip=0pt plus 1pt minus 0.1pt

\usepackage{xcolor}

%%% get total page number
\usepackage{lastpage}

%%% customized definition
\makeatletter
\def\sybtitle#1{\def\@sybtitle{#1}}
\def\sybauthor#1{\def\@sybauthor{#1}}
\def\sybdate#1{\def\@sybdate{#1}}
\sybtitle{}
\sybauthor{}
\sybdate{}
\def\sybmaketitle{
  \begin{center}
  \vspace*{.8in}
  {\huge\bfseries\@sybtitle}
  \par
  \vspace{.8in}
  {\Large\@sybauthor}
  \par
  \vspace{.2in}
  \@sybdate
  \vspace{.5in}
  \end{center}
}
\makeatother
\setlength{\parindent}{0pt}
\renewcommand{\today}{\number\month 月 \number\day 日, ~\number\year 年}
\def\lt{\textless}
\def\gt{\textgreater}
\renewcommand\contentsname{\bfseries 目~~录}
\newcommand\bs{\texttt{\symbol{'134}}} % input backslash sign
\long\def\cmd#1{\par\vspace{1em}\hspace*{6ex}#1\vspace{1em}\par}


%%% geometry
\usepackage[includehead,includefoot,hmargin=21mm,vmargin=10.5mm,
            headsep=12pt,headheight=25pt]{geometry}
%\usepackage[includehead,includefoot,hmargin=1.2in,vmargin=1in]{geometry}

%%% fancyhdr
\usepackage{fancyhdr}
\makeatletter
\fancypagestyle{main} {
  \fancyhf{} % clear header & footer
  \fancyhead[C]{\bfseries\@sybtitle}
  \fancyfoot[C]{\thepage/\pageref*{LastPage}}
  \renewcommand{\headrulewidth}{0.4pt} % header line
  \renewcommand{\footrulewidth}{0.4pt} % footer line
}
\fancypagestyle{header} {
  \fancyhf{} % clear header & footer
  \fancyfoot[C]{\Roman{page}}
  \renewcommand{\headrulewidth}{0pt} % header line
  \renewcommand{\footrulewidth}{0pt} % footer line
}
\makeatother

\usepackage{titlesec}
\titleformat{\part}{\centering\Large\bfseries}{第\,\thepart\,部分}{1em}{}
\titleformat{\section}{\large\bfseries}{\thesection}{1em}{}
\titleformat{\subsection}{\normalsize\bfseries}{\thesubsection}{1em}{}
%\titlespacing*{章节命令}{左边距}{上文距}{下文距}[右边距]
\titlespacing*{\section}{0pt}{2\baselineskip}{\parsep}


\usepackage{hyperref}

%%% perfect source code display
\usepackage{minted}
%\usemintedstyle{colorful}
\definecolor{srcbg}{rgb}{0.95,0.95,0.95}
\newminted{java}{linenos,tabsize=4,bgcolor=srcbg}
\newminted{xml}{linenos,tabsize=4,bgcolor=srcbg}
\newminted{cpp}{linenos,tabsize=4,bgcolor=srcbg}
\newminted{bash}{linenos,tabsize=4,bgcolor=srcbg}
\newminted{latex}{linenos,tabsize=4,bgcolor=srcbg}



\definetitle{\tt Android Notes}
\author{孙延宾}
\date{\today}

\begin{document}
  \tt
  \begin{titlepage}
    \maketitle
  \end{titlepage}

  \part[Commands for Android]{Android命令行工具}
  \section[aapt - Android Asset Packaging Tool]{aapt命令}
  使用方法: aapt \textless subcommand\textgreater\ \lt options\gt
  \subsection[print apk file badging information]{打印apk文件概要信息}
  aapt dump badging filename.apk

  apk文件的概括性的信息,主要用来查看versionCode、versionName信息。

  \subsection[print apk file verbose information]{打印apk文件详细信息}
  aapt list -a filename.apk

  输出的信息中包括resource和manifest两方面的内容。manifest文件中的所有信息都被打印出来,
  包括声明的Activity、Receiver、Service,各种permission、intent-filter以及
  versionCode、versionName等等,一应俱全。

  \section[pm - Package Manager]{pm命令}
  使用方法:pm \lt subcommand\gt\ \lt options\gt

  \subsection[list packages]{查看设备上的package信息}
  pm list packages [-f] [-d] [-e] [-s] [-3] FILTER

  \begin{description}
    \item[-f:] 显示package name对应的apk file路径
    \item[-d:] 只显示disabled packages
    \item[-e:] 只显示enabled packages
    \item[-s:] 只显示system packages,apk文件位于/system/app/目录
    \item[-3:] 只显示third party packages,apk文件位于/data/app/目录
    \item[FILTER:] 过滤器,如字符串"camera"将过滤出包名中包含camera的包
  \end{description}
  应用:如何删除不想用的预安装软件呢?
  \begin{enumerate}
    \item 在任务管理器中禁用不想用的应用
    \item adb shell pm list packages -f -d
    \item 删除2中的apk文件
  \end{enumerate}
  应用:使用awk批量删除不想用的预安装应用\\
  adb shell pm list packages -d | gawk -F : '\{system("adb shell pm uninstall " \$2)\}'


  \subsection[package file path]{打印apk文件路径}
  pm path \lt package-name\gt

  \subsection[enable or disable packages]{启用、禁用app}
  pm clear \lt package-name\gt

  清空package的数据信息

  pm enable \lt package-name\gt

  启用某个应用

  pm disable \lt package-name\gt

  禁用某个应用

  \subsection[安装应用]{安装应用}
  pm install [-s] [-f] [-d] [-r] \lt filename.apk\gt
  \begin{description}
    \item[-s:] 将应用安装到sdcard上
    \item[-f:] 将应用安装到内部flash上
    \item[-d:] 允许降级安装,即安装当前应用的低版本apk
    \item[-r:] reinstall app,保持应用数据不被删除
  \end{description}

  \subsection[卸载应用]{卸载应用}
  pm uninstall [-k] \lt package-name\gt

  \begin{description}
    \item[-k:] 保持应用的data和cache不被删除
  \end{description}

  \subsection[查看权限信息]{查看权限信息}
  pm list permissions [-f] [-s]

  \begin{description}
    \item[-f:] 列出所有系统权限信息
    \item[-s:] 只列出权限的摘要信息
  \end{description}


  \section[am - Application Manager]{am命令}
  am \lt subcommand\gt\ \lt options\gt

  \subsection[启动Activity]{启动Activity}
  am start [-D] [-W] [-R \lt count\gt] [-S] \lt intent\gt

  \begin{description}
    \item[-D:] 开启debug
    \item[-W:] wait for launch to complete
    \item[-R \lt count\gt:] 重复启动count次,每次启动前先结束之前的Activity
    \item[-S] 强制关闭先前的Activity再启动新的Activity
  \end{description}

  \lt intent\gt 有多种形式:

  \begin{enumerate}
    \item -a \lt ACTION\gt] [-d \lt DATA\_URI\gt] [-t \lt MIME\_TYPE\gt
    \item -c \lt CATEGORY\gt\ [-c \lt CATEGORY\gt] ...
    \item -e|--es \lt EXTRA\_KEY\gt\ \lt EXTRA\_STRING\_VALUE\gt\ ...
    \item --esn \lt EXTRA\_KEY\gt\ ...
    \item --ez \lt EXTRA\_KEY\gt\ \lt EXTRA\_BOOLEAN\_VALUE\gt\ ...
    \item --ei \lt EXTRA\_KEY\gt\ \lt EXTRA\_INT\_VALUE\gt\ ...
    \item --el \lt EXTRA\_KEY\gt\ \lt EXTRA\_LONG\_VALUE\gt\ ...
    \item --ef \lt EXTRA\_KEY\gt\ \lt EXTRA\_FLOAT\_VALUE\gt\ ...
    \item --eu \lt EXTRA\_KEY\gt\ \lt EXTRA\_URI\_VALUE\gt\ ...
    \item --ecn \lt EXTRA\_KEY\gt\ \lt EXTRA\_COMPONENT\_NAME\_VALUE\gt
    \item --eia \lt EXTRA\_KEY\gt\ \lt EXTRA\_INT\_VALUE\gt [,\lt EXTRA\_INT\_VALUE...]
    \item --ela \lt EXTRA\_KEY\gt\ \lt EXTRA\_LONG\_VALUE\gt[,\lt EXTRA\_LONG\_VALUE...]
    \item --efa \lt EXTRA\_KEY\gt\ \lt EXTRA\_FLOAT\_VALUE\gt[,\lt EXTRA\_FLOAT\_VALUE...]
    \item -n \lt COMPONENT\gt] [-f \lt FLAGS\gt
    \item --grant-read-uri-permission] [--grant-write-uri-permission]
    \item --debug-log-resolution] [--exclude-stopped-packages]
    \item --include-stopped-packages]
    \item --activity-brought-to-front] [--activity-clear-top]
    \item --activity-clear-when-task-reset] [--activity-exclude-from-recents]
    \item --activity-launched-from-history] [--activity-multiple-task]
    \item --activity-no-animation] [--activity-no-history]
    \item --activity-no-user-action] [--activity-previous-is-top]
    \item --activity-reorder-to-front] [--activity-reset-task-if-needed]
    \item --activity-single-top] [--activity-clear-task]
    \item --activity-task-on-home
    \item --receiver-registered-only] [--receiver-replace-pending]
    \item --selector
    \item \lt URI\gt\ | \lt PACKAGE\gt\ | \lt COMPONENT\gt
  \end{enumerate}

  \subsubsection[通过package name启动应用]{通过package name启动应用}
  am start -n \lt package-name\gt/\lt package-name.activity-name\gt

  如:am start -n zte.com.cn.camera/zte.com.cn.camera.Camera\\
  可以简写为:\\
  am start -n zte.com.cn.camera/.Camera

  \subsubsection[通过Intent启动应用]{通过Intent启动应用}
  am start -a android.intent.action.CALL -d tel:10086

  \subsection[启动Service]{启动Service}
  am startservice \lt intent\gt

  \subsection[关闭应用]{关闭应用}
  am force-stop \lt package-name\gt

  \subsection[发送广播]{发送广播}
  am broadcast \lt intent\gt

  \subsection[dump heap]{dump heap}
  am dumpheap \lt process\gt \lt file\gt

  \begin{description}
    \item[process:] 可以是pid,也可以是进程名称(一般为package name)
    \item[file:] 设备上的文件路径
  \end{description}
  如:am dumpheap zte.com.cn.camera /sdcard/camera.hprof

  \subsection[监视系统运行]{监视系统运行}
  am monitor

  启动、恢复app时都会有相应的package name打印。此方法可以很快得知设备上某个应用的package name.

  \subsection[设置设备的显示尺寸、显示密度]{设置设备的显示尺寸、显示密度}
  am display-size [reset|WxH]

  am display-density [reset|density]

  其中,WxH、density均为数值。

  \part[Android Notes List]{Android笔记列表}
  \section[关于package的UID]{关于package的UID}
  在安装apk时,android为application分配一个uid,每个app一个uid,除非使用sharedUid。
	uid不会随着app的启动、关闭而改变,而且重新安装似乎不会改变uid. 但是文档声明只保证
	app在device上时uid不变,如果没有package使用某个uid,该uid将会被删掉。
	查看app的uid有两个方法:
  \begin{enumerate}
    \item 查看系统的packages.xml文件(/data/system/packages.xml),从中找到app对应的package name,
		      其中便有app的uid
    \item 运行app,使用top/ps命令找到对应的pid,然后到/proc目录下查看该进程的信息:\\
      		cat /proc/\lt pid\gt/status
  \end{enumerate}

  \section[Activity的noHistory属性]{Activity的noHistory属性}
  android:noHistory="true"

  该属性置为true时,当Activity A被别的Activity B覆盖时,该Activity A会退出,
  而无法从B返回A,也就是说A不会成为Activity Stack上的“历史”Activity。
  如A正在运行,此时来电,phone结束后,A finish(),但是如果A正在运行,用户按HOME键,
  A不会finish(),因为此时A仍在栈顶。

  \section[onUserLeaveHint()]{onUserLeaveHint()}
  onUserLeaveHint()函数用于提示系统当用户离开之后该怎么做。
  举例来说:当app正在运行,此时来电,app的onPause()被执行;但是,当HOME键被按下时,
  系统会先调用onUserLeaveHint(),然后再调用onPause(),可以在该中调用finish(),
  达到HOME键退出的目的。
  
  \section[通过Intent监视SCREEN\_ON、SCREEN\_OFF动作]{通过Intent监视SCREEN\_ON、SCREEN\_OFF动作}
  这里要做的事情是:
  \begin{enumerate}
    \item 接收SCREEN\_OFF系统消息
    \item 点亮屏幕
    \item 接收SCREEN\_ON系统消息
    \item 播放视频文件(keep\_screen\_on)
  \end{enumerate}
  \subsection[定义Broadcast receiver接收Intent]{定义Broadcast receiver接收Intent}
  \inputminted[linenos,tabsize=4,bgcolor=srcbg]{java}{ScreenBroadcastReceiver.java}
  
  \subsection[在Manifest文件中声明receiver]{在Manifest文件中声明receiver}
  \inputminted[linenos,tabsize=4,bgcolor=srcbg]{xml}{AndroidManifest.xml}
    
  \subsection[使用service注册receiver]{使用service注册receiver}
  \inputminted[linenos,tabsize=4,bgcolor=srcbg]{java}{ReceiverService.java}
  
  \subsection[启动Service的桌面应用]{启动Service的桌面应用}
  \mint[bgcolor=srcbg]{java}|startService(new Intent(this, ReceiverService.class));|
  
  
  \section[如何在代码中设置View/ViewGroup的尺寸]{如何在代码中设置View/ViewGroup的尺寸}
  在代码中动态设置View的尺寸是很常见的,比如横竖屏切换时,为了保持视频画面的尺寸,
  就需要重新设置SurfaceView/TextureView/VideoView的尺寸。设置的方法是使用LayoutParams类,
  LayoutParams类有多种,这里使用的必须是:该View的父类的LayoutParams类。
  
\begin{minted}[linenos,tabsize=4,bgcolor=srcbg]{java}
LayoutParams lp = mSurfaceView.getLayoutParams();
lp.width = frameWidth;
lp.height = frameHeight;
mSurfaceView.setLayoutParams(lp);
mSurfaceView.requestLayout();
\end{minted}

  有时getLayoutParams()返回null,这时就要根据该View在XML文件中定义时,其上层对象的类型,
  定义不同LayoutParams种类。不过,应该首先尝试getLayoutParams(),因为重新定义的
  LayoutParams对象需要重新设置其他信息,比如对齐方式等,而get出来的对象都已经设置好了
  (在XML文件中)。
  
  \section[捕获所有物理按键事件]{捕获所有物理按键事件}
  \begin{minted}[linenos,tabsize=4,bgcolor=srcbg]{java}
@Override
public boolean onKeyDown(int keyCode, KeyEvent event) {
    switch(keyCode) {
        default: {
            Log.d(TAG, "keyCode: " + keyCode + ", event: " + event.getAction());
            finish();
            return true;
        }
    }
}
  \end{minted}
  
  \section[keep\_screen\_on]{keep\_screen\_on}
  \begin{minted}[linenos,tabsize=4,bgcolor=srcbg]{java}
private void keepScreenOn(boolean on) {
    if(on) {
        getWindow().addFlags(WindowManager.LayoutParams.FLAG_KEEP_SCREEN_ON);
    } else {
        getWindow().clearFlags(WindowManager.LayoutParams.FLAG_KEEP_SCREEN_ON);
    }
}
  \end{minted}
  
  \section[获取屏幕物理尺寸]{获取屏幕物理尺寸}
  \begin{minted}[linenos,tabsize=4,bgcolor=srcbg]{java}
private void getScreenSize() {
    DisplayMetrics mDisplayMetrics = new DisplayMetrics();
    getWindowManager().getDefaultDisplay().getMetrics(mDisplayMetrics);
    mScreenWidth = mDisplayMetrics.widthPixels;
    mScreenHeight = mDisplayMetrics.heightPixels;
}
  \end{minted}
  
  \section[使系统静音]{使系统静音}
  \begin{minted}[linenos,tabsize=4,bgcolor=srcbg]{java}
private void muteMusicStream(boolean mute) {
    AudioManager mAudioManager = (AudioManager) getSystemService(AUDIO_SERVICE);
    Log.d(TAG, "mute music stream: " + mute);
    if(mAudioManager != null) {
        mAudioManager.setStreamMute(AudioManager.STREAM_MUSIC, mute);
    }
}
  \end{minted}
  
  \section[使用渐变背景色模拟动画]{使用渐变背景色模拟动画}
  \begin{minted}[linenos,tabsize=4,bgcolor=srcbg]{java}
mPreviewButton = (LinearLayout) findViewById(R.id.footer);
mPreviewButton.setOnClickListener(new OnClickListener() {
    @Override
    public void onClick(View v) {
        v.setBackgroundResource(R.drawable.footer_transition);
        ((TransitionDrawable)v.getBackground()).startTransition(300);
        preview();
    }
});
  \end{minted}
  
  R.drawable.footer\_transition的定义如下:
  
  \begin{minted}[linenos,tabsize=4,bgcolor=srcbg]{xml}
<!-- drawable/footer_transition.xml -->
<?xml version="1.0" encoding="utf-8"?>
<transition xmlns:android="http://schemas.android.com/apk/res/android" >
    <item android:drawable="@color/trans_start_color" />
    <item android:drawable="@color/footer_background_color" />
</transition>
  \end{minted}

\end{document}
